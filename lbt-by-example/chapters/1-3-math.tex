\begin{lbt}
  !DRAFT
  [@META]
    TEMPLATE lbt.Doc.Chapter
    TITLE Mathematical text \textsf{(lbt.Math)}
    LABEL ch.math
    SOURCES LbtDoc, lbt.Math

  [+BODY]
    TEXT The \package{lbt.Math} template provides several affordances for typing mathematical text.
    ITEMIZE .o compact
    :: The \code{simplemath} macro is a replacement for the inline and display math environments \Verb|$ ... $| and \Verb|$$ .. $$| (or their Latex equivalents). It allows you to type mathematical text more succinctly and with fewer backslashes.
    :: The \code{integal} macro simplifies typing definite and indefinite (single) integrals.
    :: The \code{vector} and \code{vectorijk} macro greatly simplify typing vectors.
    :: The \code{MATH} command gives you easy access to \package{amsmath}'s collection of environments: align, gather, multiline, \dots
    :: A collection of macros like \code{mathlistand}, \code{mathsum}, \code{mathseqdots} and several others provide a convenient way to type mathematical text like \qq{\mathlistand{a,b,c}} or \qq{\mathseqdots{y,1,2,3,n}}.

    % ------------------------------------------------------------ The simplemath macro
    SECTION The \code{simplemath} macro

    BOX :: \verb|\lbtDefineMacros{sm = lbt.Math:simplemath}|

    TEXT
    :: The \code{simplemath} macro stands in for \Verb|$ ... $| or \Verb|$$ .. $$| and allows your to type mathematical text succinctly. It recognises a lot of keywords and abbreviations, meaning far fewer backslashes are needed.
    :: \cref{ex:math-sm-1} demonstrates the use of \code{simplemath} in both math modes: inline and display.
    :: \cref{ex:math-sm-table} contains a large number of examples.

    LBTEXAMPLE .o vertical
    :: (caption) simplemath (inline) with \Verb{\sm{...}}
    :: (label) ex:math-sm-1
    :: .v <<
      TEXT Two interesting results from Euler are \sm{ds arctan x = x - frac{x^3}3 + frac{x^5}5 - cdots} and \sm{ sum_{n=1}^{infty} frac 1 {n^2} = frac {pi^2} 6. }
    >>

    EXAMPLETABLE .o wrapcell = sm
    :: (label) ex:math-sm-table
    :: (caption) A collection of \code{simplemath} examples
    :: cos2 th + sin2 th equiv 1
    :: log_2 n ge 5
    :: ds (1+x)^n = sum_{r=0}^n binom n r x^r
    :: A = B iff A subseteq B vee B subseteq A
    :: ds al be + al ga + be ga = frac {-b} {2a}
    :: ds f'(x) = lim_{h to 0} frac {f(x+h)-f(x)} h
    :: neg(P implies Q) equiv P vee neg Q
    :: P imp Q equiv neg P wedge Q
    :: 12 = 2 cdot 2 cdot 3 text{ so } 5 nmid 12
    :: ds prod_{i=1}^n x_i = x_1 x_2 cdots x_n
    :: sqrt 2 notin bbQ
    :: forall n in bbZ, n^2 in bbZ
    :: OABC text{ is a parallelogram.}

    % ------------------------------------------------------------ The integral macro
    LATEX \FloatBarrier
    SECTION The \code{integral} macro

    TEXT
    :: Typing integrals in regular Latex, or with \code{simplemath}, is not exactly a chore. But there is room for simplification, and the \code{integral} macro allows for definite or easy typing indefinite (single, simple) integrals. If the first argument is \code{ds} then a \Verb|\displaystyle| is inserted.
    :: The resulting Latex code is wrapped in \Verb|\ensuremath{...}|, so integrals do not need to be inside a math environment.
    :: \cref{x} shows a definite and indefinite integral. \cref{y} shows the use of \code{ds} for displaystyle.

    LBTEXAMPLE .o vertical
    :: (caption) The \code{integral} macro
    :: (label) ex:math-int-1
    :: .v <<
      TEXT The integral \integral{\frac{\sin x}{x},dx}, or in display style, \integral{ds,\frac{\sin x}{x},dx}, has practical importance but can't be evaluated in closed form.

      TEXT You can take advantage of \code{simplemath} as well: \sm{ \integral{sqrt {1 + sin3 th},d th} }
    >>

    EXAMPLETABLE
    :: (label) ex:math-int-table-1
    :: (caption) A collection of \code{integral} examples, both definite and indefinite, with and without displaystyle.
    :: (shortcaption) A collection of \code{integral} examples
    :: \integral{\sin x,dx}
    :: \integral{1,3,\sqrt{4z},dz}
    :: \integral{ds,\tan y,dy}
    :: \integral{ds,1,3,\sqrt{4z},dz}



    % ------------------------------------------------------------ The vector macro and friends
    LATEX \FloatBarrier
    SECTION The \code{vector} macro

    TEXT
    :: With the \code{vector} macro you can easily typeset \vecbold{a} and \vectilde{b} and \vecarrow{c}. Oh, and \V{DE} and \V{3 -7} and \V{row 4 0 9}. And finally, \V{ijk 2 -5 1} and \V{ijk -3 0 4}.
    :: One might choose to set up this macro as \Verb|\V| as follows:
    BOX \verb|\lbtDefineMacros{V = lbt.Math:vector}|
    TEXT A thorough set of examples is given in \cref{ex:math-vector-table}.

    EXAMPLETABLE .o mathmode
    :: (label) ex:math-vector-table
    :: (caption) A collection of \code{vector} and \code{vectorijk} examples
    :: \V{3 1 -9} + \V{-2 5 4} = \V{1 6 -5}
    :: \V{row 2 7 -1 4 6}
    :: \V{a} + \V{b} = \V{a_1 a_2} + \V{b_1 b_2}
    :: \V{r} = \V{r_1 \vdots r_n}
    :: \V{3 1 -9} = \Vijk{3 1 -9}
    :: \V{-1 0 4} = \Vijk{-1 0 4}
    :: \V{-1 0} = \Vijk{-1 0}
    :: \V{ijk 3 4 5}

    SUBSECTION Bold, tilde and arrow
    TEXT Vectors are most commonly set in bold upright (\vecbold{a}), and that is the \lbtlogo{} default. You can, however, choose tilde (\vectilde{a}) or arrow (\vecarrow{a}) instead. \cref{ex:math-vector-formats-1} shows how to make these choices for your \lbtlogo{} expansion or your whole Latex document. If you want just a one-off, you can define and use the following macros. \cref{ex:math-vector-formats-2} demonstrates their use.
    BOX
    :: \verb|\lbtDefineMacros{vecbold  = lbt.Math:vecbold}|
    :: \verb|\lbtDefineMacros{vectilde = lbt.Math:vectilde}|
    :: \verb|\lbtDefineMacros{vecarrow = lbt.Math:vecarrow}|

    LBTEXAMPLE .o output = 0
    :: (caption) Vectors in bold or tilde or arrow format, document-wide
    :: (label) ex:math-vector-formats-1
    :: (substitute) LBT/lbt
    :: .v <<
      * Affect the whole Latex document
      \lbtGlobalOpargs{vector.format = tilde}

      * Affect one LBT expansion
      \begin{LBT}
        [@META]
          TEMPLATE lbt.Basic
          OPTIONS  vector.format = arrow
        ...
      \end{LBT}
    >>

    LBTEXAMPLE .o vertical
    :: (caption) Vectors in bold or tilde or arrow format, one-off
    :: (label) ex:math-vector-formats-2
    :: .v <<
      TEXT We can write \vecbold{a} or \vectilde{b} or \vecarrow{c}.
    >>



    % ------------------------------------------------------------ The MATH command
    SECTION The \code{MATH} command

    % ------------------------------------------------------------ The mathlistand macro and friends
    SECTION The \code{mathlistand} macro and friends

\end{lbt}
