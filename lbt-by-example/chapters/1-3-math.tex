\begin{lbt}
  !DRAFT
  [@META]
    TEMPLATE lbt.Doc.Chapter
    TITLE Mathematical text \textsf{(lbt.Math)}
    LABEL ch.math
    SOURCES LbtDoc, lbt.Math

  [+BODY]
    TEXT The \package{lbt.Math} template provides several affordances for typing mathematical text.
    ITEMIZE .o compact
    :: The \code{simplemath} macro is a replacement for the inline and display math environments \Verb|$ ... $| and \Verb|$$ .. $$| (or their Latex equivalents). It allows you to type mathematical text more succinctly and with fewer backslashes.
    :: The \code{integal} macro simplifies typing definite and indefinite (single) integrals.
    :: The \code{vector} and \code{vectorijk} macro greatly simplify typing vectors.
    :: The \code{MATH} command gives you easy access to \package{amsmath}'s collection of environments: align, gather, multiline, \dots
    :: A collection of macros like \code{mathlistand}, \code{mathsum}, \code{mathseqdots} and several others provide a convenient way to type mathematical text like \qq{\mathlistand{a,b,c}} or \qq{\mathseqdots{y,1,2,3,n}}.

    % ------------------------------------------------------------ The simplemath macro
    SECTION The \code{simplemath} macro

    BOX :: \verb|\lbtDefineMacros{sm = lbt.Math:simplemath}|

    TEXT
    :: The \code{simplemath} macro stands in for \Verb|$ ... $| or \Verb|$$ .. $$| and allows your to type mathematical text succinctly. It recognises a lot of keywords and abbreviations, meaning far fewer backslashes are needed. For example, you can type \Verb|cos^2 th = 0.25 implies cos th = pm 0.5| instead of \Verb|\cos^2 \theta = 0.25 \Longrightarrow \cos \theta = \pm 0.5|.
    :: \cref{ex:math-sm-1} demonstrates some keywords (\code{sin}, \code{cos}, \code{equiv}, \code{sum}, \code{frac}, \code{pi}) and some abbreviations \code{al} and \code{th} (for \verb|\alpha| and \verb|\theta|). Also \code{ds} for \verb|\displaystyle|.
    :: \cref{ex:math-sm-2} demonstrates several keywords: \code{forall}, \code{exists}, \code{in}, \code{implies} and \code{text}.

    EXAMPLETABLE .o wrapcell = sm
    :: (label) ex:math-sm-table
    :: (caption) A collection of \code{simplemath} examples
    :: cos2 th + sin2 th equiv 1
    :: log_2 n ge 5
    :: ds (1+x)^n = sum_{r=0}^n binom n r x^r
    :: A = B iff A subseteq B vee B subseteq A
    :: ds al be + al ga + be ga = frac {-b} {2a}
    :: ds f'(x) = lim_{h to 0} frac {f(x+h)-f(x)} h
    :: neg(P implies Q) equiv P vee neg Q
    :: P imp Q equiv neg P wedge Q
    :: 12 = 2 cdot 2 cdot 3 text{ so } 5 nmid 12
    :: ds prod_{i=1}^n x_i = x_1 x_2 cdots x_n
    :: sqrt 2 notin bbQ
    :: forall n in bbZ, n^2 in bbZ
    :: OABC text{ is a parallelogram.}

    LBTEXAMPLE .o vertical
    :: (caption) simplemath (inline) with \Verb{\sm{...}}
    :: (label) ex:math-sm-1
    :: .v <<
      ITEMIZE .o compact
      :: \sm{cos th = 0.27}
      :: \sm{cos^2 al + sin^2 al equiv 1}
      :: \sm{ds sum_{n=1}^{infty} frac 1 {n^2} = frac {pi^2} 6}
    >>

    LBTEXAMPLE .o vertical
    :: (caption) simplemath (display) with \Verb{\sm{ ... }} (note spaces)
    :: (label) ex:math-sm-2
    :: .v <<
      TEXT The \emph{intermediate value theorem}: \sm{ forall f text{ continuous on } [a,b], f(a) < 0 < f(b) implies exists c in (a,b) text{ such that } f(c) = 0 }
    >>

    % ------------------------------------------------------------ The integral macro
    SECTION The \code{integral} macro

    % ------------------------------------------------------------ The vector macro and friends
    SECTION The \code{vector} macro and friends

    % ------------------------------------------------------------ The MATH command
    SECTION The \code{MATH} command

    % ------------------------------------------------------------ The mathlistand macro and friends
    SECTION The \code{mathlistand} macro and friends

\end{lbt}
