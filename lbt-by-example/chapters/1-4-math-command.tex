\begin{lbt}
  !DRAFT
  [@META]
    TEMPLATE lbt.Doc.Chapter
    TITLE Mathematical text \textsf{(lbt.Math)} -- the \code{MATH} command
    LABEL ch.math-command
    SOURCES LbtDoc

  [+BODY]
    STO lbt :: 999 :: \lbtlogo{}
    STO MATH :: 999 :: \code{MATH}
    STO amsmath :: 999 :: \package{amsmath}
    STO mathtools :: 999 :: \package{amsmath}
    STO split :: 999 :: \code{split}
    STO eqsplit :: 999 :: \code{eqsplit}
    STO equation :: 999 :: \code{equation}
    STO align :: 999 :: \code{align}
    STO gather :: 999 :: \code{gather}
    STO multline :: 999 :: \code{multline}
    STO simplemath :: 999 :: \code{simplemath}
    STO xxx :: 999 :: \code{xxx}
    STO xxx :: 999 :: \code{xxx}

    TEXT
    :: The ◊MATH command gets its own chapter so that its various features can be displayed one section at a time.
    :: ◊MATH provides for a variety of display equations. It is a portal to various \package{amsmath} and \package{mathtools} environments like ◊split, ◊gather, ◊align, and so on. The examples here give a good primer on their use, but readers should consult the relevant documentation to develop greater awareness of the details.

    % ------------------------------------------------------------ Opening remarks
    SECTION Opening remarks

    STO fn1 :: 1 :: Or the Tex command \Verb|$$ ... $$|, which is lower-level and may produce different vertical spacing from \Verb|\[ ... \]|.

    TEXT
    :: Setting a display equation with \Verb|\[ ... \]|\footnote{◊fn1} is enough for a great many cases. If you want your equation to be numbered, you upgrade to the \code{equation} environment. If the math content to be displayed is more complicated than that, the author should decide which of the following applies:
    ITEMIZE .o compact
    :: there is one logical equation with several parts (separated by $=$ or $>$ or \dots) that should appear on separate lines (\code{split});
    :: there is one logical equation that is too long to fit on one line (\code{multline});
    :: there are several logical equations to be displayed together (\code{gather});
    :: there are several logical equations to be displayed reasonably with alignment (\code{align});
    :: there are more complicated alignment requirements, perhaps involving comments to the side (also \code{align}).
    TEXT
    :: Based on that, the author can choose an ◊amsmath environment, which are demonstrated in \cref{table:amsmath-envs}. The table does not show \emph{all} available environments, but it gives a good overview for readers who are not already familiar.
    :: The sections of this chapter give more detailed information on these environments and more.

    NOTE In normal Latex code, equations are numbered by default. If you use the \code{align} environment then all lines are numbered. If you use \code{align*} then none of them are. \par \indent ◊lbt is similar: use \code{MATH} to get numbered equations and \code{MATH*} to suppress numbereding. If you want unnumbered equations by default, set the oparg \code{MATH.eqnum = false}.

    TEXT The ◊lbt examples that follow will demonstrate fully numbered, partially numbered, and unnumbered equations, as appropriate to the environment being demonstrated.

    % NEWCOMMAND amsexample :: 2 :: \begin{minipage}{7cm}{\begin{#1} #2 \end{#1}}
    NEWCOMMAND amsexample :: 2 :: \begin{minipage}{\linewidth}{\setcounter{equation}{0}\begin{#1} #2 \end{#1}}\end{minipage}

    STO splitexample :: 1 :: \begin{split}f'(x) &= \lim_{h\to0} \frac{f(x+h) - f(x)}{h} \\ &= \lim_{h\to0} \frac{(x+h)^2 - x^2}{h} \\ &= \lim_{h\to0} \frac{(x^2 + 2xh + h^2) - x^2}{h} \\ &= \lim_{h\to0} \frac{2xh + h^2}{h} \\ &= \lim_{h\to0} 2x + h \\ &= 2x\end{split}
    STO half :: 1 :: \frac{1}{2}

    TABLE .o float
    :: (spec) colspec = {lX}, cells = {font=\small}, row{1} = {font=\bfseries\small}, cell{1}{2} = {halign=c}
    :: (caption) Some environments provided by ◊amsmath
    :: (label) table:amsmath-envs
    :: Environment & Example
    :: \hline
    :: ◊equation & \amsexample{equation}{a^2 + b^2 = c^2}
    :: ◊gather   & \amsexample{gather}{a^2 + b^2 = c^2 \\ E = mc^2 \\ F = k \frac{q_1q_2}{r^2}}
    :: ◊align (1)   & \amsexample{align}{a^2 + b^2 &= c^2 \\ E &= mc^2 \\ F &= k \frac{q_1q_2}{r^2}}
    :: ◊align (2)    & \amsexample{align}{a^2 + b^2 &= c^2  &  E &= mc^2 \\ F &= k \frac{q_1q_2}{r^2}  &  F &= ma}
    :: ◊align (3)    & \amsexample{align}{2^{n+1} &= 2\cdot 2^n \\ &> 2\cdot n^2 &&\text{by assumption} \\ &= n^2 + ◊half n^2 + ◊half n^2 \\ &> n^2 + 2n + 1 &&\text{reader to confirm} \\ &= (n+1)^2}
    :: ◊split {\footnotesize(inside ◊equation)} & \amsexample{equation}{◊splitexample}
    :: ◊multline & \amsexample{multline}{(1+x)^n = \sum_{r=0}^n \binom{n}{r}\,x^r = 1 + \binom n1 x + \binom n2 x^2 \\ + \dots + \binom nr x^r + \dots + \binom nn x^n}

    % ------------------------------------------------------------ equation
    SECTION Simple equations with \code{MATH}

    TEXT
    :: The ◊equation environment provides for a simple numbered equation. \cref{ex:math-equation} demonstrates this in ◊lbt. ◊equation is in fact the default environment, so you can just write \code{MATH F = ma}, as the example shows.

    LBTEXAMPLE .o horizontal, reseteqnum
    :: (label) ex:math-equation
    :: (caption) \code{MATH .o equation} to format a simple equation
    :: .v <<
      TEXT Newton's second law is known to many.
      MATH F = ma

      TEXT You can suppress numbering in two ways.
      MATH* F = ma
      MATH .o noeqnum :: F = ma

      TEXT \code{equation} is the \emph{default} environment for \code{MATH}, but you can be explicit if you wish.
      MATH .o equation :: F = ma
    >>

    % ------------------------------------------------------------ multiline
    LATEX \FloatBarrier
    SECTION Long equations with \code{MATH .o multiline}

    TEXT
    :: If an equation is too long for one line, you can insert linebreaks and the \package{amsmath} environment ◊multline will handle formatting with a mixture of left, center and right justification. \cref{ex:math-multiline} demonstrates.
    :: Both \code{MATH .o multiline} and \code{MATH .o multline} work, and do the same thing.
    :: Note that the example includes the oparg \code{sm = false} to disable ◊simplemath. This is necessary to prevent \code{be} from being rendered as \sm{be}.

    LBTEXAMPLE .o vertical, position = htbp
    :: (label) ex:math-multline-1
    :: (caption) A long equation with \code{multiline}
    :: .v <<
      TEXT The display environment below has its margins adjusted so that the effect of \code{multiline} can best be seen.

      MATH .o multiline, sm = false, adjustwidth = 2cm 2cm
      :: (a+b+c+d+e)^2 =
      :: a^2 + 2ab + 2ac + 2ad + 2ae + b^2 + 2bc + 2bd + 2be
      :: + c^2 + 2cd + 2ce + d^2 + 2de + e^2
    >>

    % ------------------------------------------------------------ eqsplit
    LATEX \FloatBarrier
    SECTION ◊eqsplit

    TEXT
    :: The ◊amsmath environment ◊split is designed for a single logical equation that is broken into two or more lines, like the example below.
    MATH .o eqsplit
      :: (a+b)^2 &= (a+b)(a+b)
      ::         &= a^2 + ab + ab + b^2
      ::         &= a^2 + 2ab + b^2
    TEXT
    :: However, ◊split is a sub-environment that can only occur within a display environment like ◊equation or ◊gather or ◊align. The most common use would be for a ◊split appear alone in an ◊equation environment, so ◊lbt provides \code{MATH .o eqsplit} for that purpose.
    :: An ◊eqsplit equation gets one number overall, not one number per line, because it is one logical equation. It is the ◊equation environment that provides the number, not the contained ◊split.
    :: \cref{ex:math-eqsplit-1} demonstrates an unnumbered split equation that is aligned on $=$ and~$>$. \cref{ex:math-eqsplit-2} shows a split equation that is numbered and referenced.

    LBTEXAMPLE .o vertical, position = htbp
    :: (label) ex:math-eqsplit-1
    :: (caption) Using \code{MATH .o eqsplit} for a multi-step equation
    :: .v <<
      TEXT Part of a proof by induction.
      STO half :: 1 :: $\tfrac 1 2$
      MATH* .o eqsplit
      :: 2^{n+1} &= 2 cdot 2^n
      ::         &> 2 cdot n^2
      ::         &= n^2 + ◊half n^2 + ◊half n^2
      ::         &> n^2 + 2n + 1
      ::         &= (n+1)^2
    >>

    TEXT Note that it could be desirable to add aligned comments off to the right in \cref{ex:math-eqsplit-1}. Unfortunately that is not possible with ◊eqsplit (or the underlying ◊split). Later, in \cref{sec:math-align}, this example will be revisited. The problem there is that ◊align gives one number per line, because it sees each line as a separate equation.

    LBTEXAMPLE .o vertical, position = htbp
    :: (label) ex:math-eqsplit-2
    :: (caption) A \code{MATH .o eqsplit} equation that is referenced
    :: .v <<
      MATH .o eqsplit, label = eq1
      :: (a+b)^2 &= (a+b)(a+b)
      ::         &= a^2 + ab + ab + b^2
      ::         &= a^2 + 2ab + b^2

      TEXT As shown in \eqref{eq1}, $(a+b)^2$ does not equal $a^2 + b^2$!
    >>


    LBTEXAMPLE .o vertical
    :: (label) ex:math-eqsplit-2
    :: (caption) Using \code{eqsplit} to improve the formatting of a long equation
    :: .v <<
      MATH* .o eqsplit
      :: (x - r_1)(x - r_2)(x - r_3)(x - r_4)
      :: quad &= x^4 - (r_1 + r_2 + r_3 + r_4)\,x^3
      ::      &+ (r_1r_2 + r_1r_3 + r_1r_4 + r_2r_3 + r_2r_4 + r_3r_4)\,x^2
      ::      &- (r_1r_2r_3 + r_1r_2r_4 + r_1r_3r_4 + r_2r_3r_4)\,x
      ::      &+ r_1r_2r_3r_4
    >>

    % ------------------------------------------------------------ align
    LATEX \FloatBarrier
    SECTION (label) sec:math-align :: \code{align}

    % ------------------------------------------------------------ gather
    LATEX \FloatBarrier
    SECTION \code{gather}



    % ------------------------------------------------------------ Other environments
    LATEX \FloatBarrier
    SECTION (label) sec:math-other :: \code{Other environments}

    SUBSECTION ◊split
    TEXT
    :: \textbf{Revisit this text in light of it being in the \qq{other} section}
    :: ◊MATH provides the \code{split} option to access the ◊split environment, but it is not likely to be all that useful, because of the need to enclose it in another environment. The example below shows the ◊lbt code and resulting Latex code.

    LBTEXAMPLE .o float = false, horizontal, output = 1, shrinkmargin = nil
    :: .v <<
      MATH .o split
      :: (a+b)^2 &= (a+b)(a+b)
      ::         &= a^2 + ab + ab + b^2
      ::         &= a^2 + 2ab + b^2
    >>




    % ------------------------------------------------------------ xxx
    LATEX \FloatBarrier
    SECTION \code{xxx}



    % ------------------------------------------------------------ Summary
    LATEX \FloatBarrier
    SECTION \code{Summary of the \code{MATH} command}





\end{lbt}
