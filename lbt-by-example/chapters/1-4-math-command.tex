\begin{lbt}
  !DRAFT
  [@META]
    TEMPLATE lbt.Doc.Chapter
    OPTIONS LBTEXAMPLE.position = htbp
    TITLE Mathematical text \textsf{(lbt.Math)} -- the \code{MATH} command
    LABEL ch.math-command
    SOURCES LbtDoc

  [+BODY]
    STO lbt :: 999 :: \lbtlogo{}
    STO MATH :: 999 :: \code{MATH}
    STO amsmath :: 999 :: \package{amsmath}
    STO mathtools :: 999 :: \package{mathtools}
    STO split :: 999 :: \code{split}
    STO eqsplit :: 999 :: \code{eqsplit}
    STO equation :: 999 :: \code{equation}
    STO multiline :: 999 :: \code{multiline}
    STO align :: 999 :: \code{align}
    STO alignat :: 999 :: \code{alignat}
    STO alignedat :: 999 :: \code{alignedat}
    STO eqalignedat :: 999 :: \code{eqalignedat}
    STO flalign :: 999 :: \code{flalign}
    STO gather :: 999 :: \code{gather}
    STO gathered :: 999 :: \code{gathered}
    STO lgathered :: 999 :: \code{lgathered}
    STO rgathered :: 999 :: \code{rgathered}
    STO eqgathered :: 999 :: \code{eqgathered}
    STO eqlgathered :: 999 :: \code{eqlgathered}
    STO eqrgathered :: 999 :: \code{eqrgathered}
    STO multline :: 999 :: \code{multline}
    STO simplemath :: 999 :: \code{simplemath}
    STO continuation :: 999 :: \code{continuation}
    STO commentary :: 999 :: \code{commentary}
    STO xxx :: 999 :: \code{xxx}
    STO xxx :: 999 :: \code{xxx}

    TEXT
    :: The ◊MATH command gets its own chapter so that its various features can be displayed one section at a time.
    :: ◊MATH provides for a variety of display equations. It is a portal to various \package{amsmath} and \package{mathtools} environments like ◊split, ◊gather, ◊align, and so on. The examples here give a good primer on their use, but readers should consult the relevant documentation to develop greater awareness of the details.

    % ------------------------------------------------------------ Opening remarks
    SECTION Opening remarks

    STO fn1 :: 1 :: Or the Tex command \Verb|$$ ... $$|, which is lower-level and may produce different vertical spacing from \Verb|\[ ... \]|.

    TEXT
    :: Setting a display equation with \Verb|\[ ... \]|\footnote{◊fn1} is enough for a great many cases. If you want your equation to be numbered, you upgrade to the \code{equation} environment. If the math content to be displayed is more complicated than that, the author should decide which of the following applies:
    ITEMIZE .o compact
    :: there is one logical equation with several parts (separated by $=$ or $>$ or \dots) that should appear on separate lines (\code{split});
    :: there is one logical equation that is too long to fit on one line (\code{multline});
    :: there are several logical equations to be displayed together (\code{gather});
    :: there are several logical equations to be displayed reasonably with alignment (\code{align});
    :: there are more complicated alignment requirements, perhaps involving comments to the side (also \code{align}).
    TEXT
    :: Based on that, the author can choose an ◊amsmath environment, which are demonstrated in \cref{table:amsmath-envs}. The table does not show \emph{all} available environments, but it gives a good overview for readers who are not already familiar.
    :: The sections of this chapter give more detailed information on these environments and more.

    NOTE In normal Latex code, equations are numbered by default. If you use the \code{align} environment then all lines are numbered. If you use \code{align*} then none of them are. \par \indent ◊lbt is similar: use \code{MATH} to get numbered equations and \code{MATH*} to suppress numbereding. If you want unnumbered equations by default, set the oparg \code{MATH.eqnum = false}.

    TEXT The ◊lbt examples that follow will demonstrate fully numbered, partially numbered, and unnumbered equations, as appropriate to the environment being demonstrated.

    % NEWCOMMAND amsexample :: 2 :: \begin{minipage}{7cm}{\begin{#1} #2 \end{#1}}
    NEWCOMMAND amsexample :: 2 :: \begin{minipage}{\linewidth}{\setcounter{equation}{0}\begin{#1} #2 \end{#1}}\end{minipage}

    STO splitexample :: 1 :: \begin{split}f'(x) &= \lim_{h\to0} \frac{f(x+h) - f(x)}{h} \\ &= \lim_{h\to0} \frac{(x+h)^2 - x^2}{h} \\ &= \lim_{h\to0} \frac{(x^2 + 2xh + h^2) - x^2}{h} \\ &= \lim_{h\to0} \frac{2xh + h^2}{h} \\ &= \lim_{h\to0} 2x + h \\ &= 2x\end{split}
    STO half :: 1 :: \frac{1}{2}

    TABLE .o float
    :: (spec) colspec = {lX}, cells = {font=\small}, row{1} = {font=\bfseries\small}, cell{1}{2} = {halign=c}
    :: (caption) Some environments provided by ◊amsmath
    :: (label) table:amsmath-envs
    :: Environment & Example
    :: \hline
    :: ◊equation & \amsexample{equation}{a^2 + b^2 = c^2}
    :: ◊gather   & \amsexample{gather}{a^2 + b^2 = c^2 \\ E = mc^2 \\ F = k \frac{q_1q_2}{r^2}}
    :: ◊align (1)   & \amsexample{align}{a^2 + b^2 &= c^2 \\ E &= mc^2 \\ F &= k \frac{q_1q_2}{r^2}}
    :: ◊align (2)    & \amsexample{align}{a^2 + b^2 &= c^2  &  E &= mc^2 \\ F &= k \frac{q_1q_2}{r^2}  &  F &= ma}
    :: ◊align (3)    & \amsexample{align}{2^{n+1} &= 2\cdot 2^n \\ &> 2\cdot n^2 &&\text{by assumption} \\ &= n^2 + ◊half n^2 + ◊half n^2 \\ &> n^2 + 2n + 1 &&\text{reader to confirm} \\ &= (n+1)^2}
    :: ◊split {\footnotesize(inside ◊equation)} & \amsexample{equation}{◊splitexample}
    :: ◊multline & \amsexample{multline}{(1+x)^n = \sum_{r=0}^n \binom{n}{r}\,x^r = 1 + \binom n1 x + \binom n2 x^2 \\ + \dots + \binom nr x^r + \dots + \binom nn x^n}

    % ------------------------------------------------------------ equation
    LATEX \FloatBarrier
    SECTION Simple equations with \code{MATH (.o equation)}

    TEXT
    :: The ◊equation environment provides for a simple numbered equation. \cref{ex:math-equation} demonstrates this in ◊lbt. ◊equation is in fact the default environment, so you can just write \code{MATH F = ma}, as the example shows.

    LBTEXAMPLE .o horizontal, reseteqnum
    :: (label) ex:math-equation
    :: (caption) \code{MATH .o equation} to format a simple equation
    :: .v <<
      TEXT Newton's second law is known to many.
      MATH F = ma

      TEXT You can suppress numbering in two ways.
      MATH* F = ma
      MATH .o noeqnum :: F = ma

      TEXT \code{equation} is the \emph{default} environment for \code{MATH}, but you can be explicit if you wish.
      MATH .o equation :: F = ma
    >>

    % ------------------------------------------------------------ multiline
    LATEX \FloatBarrier
    SECTION Long equations with \code{MATH .o multiline}

    TEXT
    :: If an equation is too long for one line, you can insert linebreaks and the \package{amsmath} environment ◊multline will handle formatting with a mixture of left, center and right justification. \cref{ex:math-multiline} demonstrates.
    :: Both \code{MATH .o multiline} and \code{MATH .o multline} work, and do the same thing.
    :: Note that the example includes the oparg \code{sm = false} to disable ◊simplemath. This is necessary to prevent \code{be} from being rendered as \sm{be}.

    LBTEXAMPLE .o vertical, position = htbp
    :: (label) ex:math-multiline
    :: (caption) A long equation with ◊multiline
    :: .v <<
      TEXT The display environment below has its margins adjusted so that the effect of ◊multiline

      MATH .o multiline, sm = false, adjustwidth = 2cm 2cm
      :: (a+b+c+d+e)^2 =
      :: a^2 + 2ab + 2ac + 2ad + 2ae + b^2 + 2bc + 2bd + 2be
      :: + c^2 + 2cd + 2ce + d^2 + 2de + e^2
    >>

    % ------------------------------------------------------------ gather
    LATEX \FloatBarrier
    SECTION \code{gather}

    TEXT
    :: If you have a few equations that you want to display at once, and aren't too fussed about how they are presented, then you ◊gather them and they will each be centered and numbered. Two examples suffice. \cref{ex:math-gather-1} demonstrates the default numbering. \cref{ex:math-gather-2} demonstrates selective numbering. If you want no numbering, just use \code{MATH*}.
    :: Note that the first example (\cref{ex:math-gather-1}) demonstrates the oparg \code{linespace}, which provides some extra space between the lines.

    LBTEXAMPLE .o vertical
    :: (label) ex:math-gather-1
    :: (caption) Displaying several equations with ◊gather
    :: .v <<
      TEXT Some interesting series results.
      MATH .o gather, linespace = 1ex
      % :: sum_{k=1}^n k = \tfrac12 n (n+1)
      :: sum_{n=1}^infty frac 1 {n^2} = frac {pi^2} 6
      :: sum_{n=0}^infty frac {x^n} {n!} = e^x
    >>

    LBTEXAMPLE .o vertical
    :: (label) ex:math-gather-2
    :: (caption) ◊gather with selective numbering
    :: .v <<
      TEXT Area formulas you must know:
      MATH .o gather, eqnum = 2 4
      :: A = pi r ^2
      :: A = pi a b       \label{eq:a1}
      :: A = \tfrac12 b h
      :: A = \tfrac12 x y   \label{eq:a2}
      TEXT We'll discuss \eqref{eq:a1} and \eqref{eq:a2} in more detail\dots
    >>

    SUBSECTION Building-block variants of ◊gather

    TEXT
    :: Using ◊gathered instead of ◊gather, you can gather some equations into one logical object, and place that inside of ◊equation (or another display environment) to assign an equation number. In addition to ◊amsmath's ◊gathered, ◊mathtools provides ◊lgathered and ◊rgathered, which justify the equations left or right within the centered block.
    :: ◊lbt provides ◊eqgathered and ◊eqlgathered and ◊eqrgathered as composite environments which wrap the building-block inside ◊equation. These are likely to be of only niche interest. \cref{ex:math-gather-3} and \cref{ex:math-gather-4} demonstrate these environments. Note the way they assign only one equation number.

    LBTEXAMPLE .o vertical
    :: (label) ex:math-gather-3
    :: (caption) ◊eqgathered: a single equation number for a gathered block
    :: .v <<
      MATH .o eqgathered
      :: sin x = text{opp} / text{hyp} quad text{and} quad cos x = text{adj} / text{hyp}
      :: sin2x + cos2x equiv 1
    >>

    LBTEXAMPLE .o vertical
    :: (label) ex:math-gather-4
    :: (caption) The left/right justified variants of ◊eqgathered
    :: .v <<
      MATH .o eqlgathered
      :: sin x = text{opp} / text{hyp} quad text{and} quad cos x = text{adj} / text{hyp}
      :: sin2x + cos2x equiv 1
      MATH .o eqrgathered
      :: sin x = text{opp} / text{hyp} quad text{and} quad cos x = text{adj} / text{hyp}
      :: sin2x + cos2x equiv 1
    >>


    % ------------------------------------------------------------ eqsplit
    LATEX \FloatBarrier
    SECTION (label) sec:eqsplit
    :: Several-part equations with ◊eqsplit

    TEXT
    :: The ◊amsmath environment ◊split is designed for a single logical equation that is broken into two or more lines, like the example below.
    MATH .o eqsplit
      :: (a+b)^2 &= (a+b)(a+b)
      ::         &= a^2 + ab + ab + b^2
      ::         &= a^2 + 2ab + b^2
    TEXT
    :: However, ◊split is a sub-environment that can only occur within a display environment like ◊equation or ◊gather or ◊align. The most common use would be for a ◊split appear alone in an ◊equation environment, so ◊lbt provides \code{MATH .o eqsplit} for that purpose.
    :: An ◊eqsplit equation gets one number overall, not one number per line, because it is one logical equation. It is the ◊equation environment that provides the number, not the ◊split environment contained within.
    :: \cref{ex:math-eqsplit-1} demonstrates an unnumbered split equation that is aligned on $=$ and~$>$. \cref{ex:math-eqsplit-2} shows a split equation that is numbered and referenced.

    LBTEXAMPLE .o horizontal, position = htbp
    :: (label) ex:math-eqsplit-1
    :: (caption) Using \code{MATH .o eqsplit} for a multi-step equation
    :: .v <<
      TEXT Part of a proof by induction.
      STO half :: 1 :: $\tfrac 1 2$
      MATH* .o eqsplit
      :: 2^{n+1} &= 2 cdot 2^n
      ::         &> 2 cdot n^2
      ::      &= n^2 + ◊half n^2 + ◊half n^2
      ::         &> n^2 + 2n + 1
      ::         &= (n+1)^2
    >>

    TEXT Note that it could be desirable to add aligned comments to the right of each line in \cref{ex:math-eqsplit-1}. Unfortunately that is not possible with ◊eqsplit (or the underlying ◊split). Later, in \cref{sec:math-align}, this example will be revisited using the more flexible ◊align. However, ◊align emits one equation number per line, because it sees each line as a separate equation. So it is not a perfect match for providing side-by-side commentary, but it can be adapted to the task.

    LBTEXAMPLE .o horizontal, position = htbp
    :: (label) ex:math-eqsplit-2
    :: (caption) A \code{MATH .o eqsplit} equation that is referenced
    :: .v <<
      MATH .o eqsplit, label = eq1
      :: (a+b)^2 &= (a+b)(a+b)
      ::         &= a^2 + ab + ab + b^2
      ::         &= a^2 + 2ab + b^2

      TEXT As shown in \eqref{eq1}, $(a+b)^2$ does not equal $a^2 + b^2$!
    >>


    % ------------------------------------------------------------ align
    LATEX \FloatBarrier
    SECTION (label) sec:math-align
    :: \code{align} and \code{alignat} and \code{flalign}

    TEXT
    :: The \package{amsmath} environments ◊align and its variants are supported directly by \code{MATH .o align} and friends to produce mathematical output using specified alignment points. Note that ◊split (\cref{sec:eqsplit}) does this as well, but it only allows one alignment point per line. ◊align and ◊alignat allow as many alignment points as you wish.
    :: The difference between ◊align and the others is as follows:
    ITEMIZE .o compact
    :: ◊align determines the horizontal spacing between alignment columns itself, while centering the material overall;
    :: ◊flalign (\qq{full-length align}) spreads the columns out as far as possible, using the full text width of the page;
    :: ◊alignat requires the author to specify the number of columns and to control the spacing between them.
    TEXT There are different logical uses for the alignment environments, as the subsections below demonstrate.

    % ------------------------------------------------------------ align: 1
    LATEX \FloatBarrier
    SUBSECTION Presenting a group of equations with simple alignment

    TEXT
    :: The mathematical content in \cref{ex:math-align-1}, \cref{ex:math-align-2} and \cref{ex:math-align-3} is the same, but different numbering choices are demonstrated.
    :: These examples use only one alignment point, so the material could technically be typeset by ◊eqsplit. This would be the wrong logical choice, however, because these are three equations, not one. Accordingly, ◊align produces (up to) three equation numbers whereas ◊eqsplit would only produce one.

    LBTEXAMPLE .o vertical
    :: (label) ex:math-align-1
    :: (caption) Aligning a group of equations with ◊align
    :: .v <<
      TEXT All equations numbered (the default).
      MATH .o align
      ::    (a+b)^3 &= a^3 + 3a^2b + 3ab^2 + b^3
      :: (a-b)(a+b) &= a^2 - b^2
      ::        c^2 &= a^2 + b^2
    >>

    LBTEXAMPLE .o vertical
    :: (label) ex:math-align-2
    :: (caption) Alignment without numbering
    :: .v <<
      TEXT Numbering suppressed.
      MATH* .o align
      ::    (a+b)^3 &= a^3 + 3a^2b + 3ab^2 + b^3
      :: (a-b)(a+b) &= a^2 - b^2
      ::        c^2 &= a^2 + b^2
    >>

    TEXT \cref{ex:math-align-3} shows a special feature of ◊MATH: selective numbering. In ordinary Latex, you use \Verb|\notag| on any line you do not want numbered. (You can do that in ◊MATH too if you wish.) The ◊MATH oparg \code{eqnum} gives you convenient control over which lines are numbered, without editing the lines themselves.

    LBTEXAMPLE .o vertical
    :: (label) ex:math-align-3
    :: (caption) Alignment with selective numbering
    :: .v <<
      TEXT Selective numbering.
      MATH .o align, eqnum = 1 3
      ::    (a+b)^3 &= a^3 + 3a^2b + 3ab^2 + b^3
      :: (a-b)(a+b) &= a^2 - b^2
      ::        c^2 &= a^2 + b^2
    >>

    % ------------------------------------------------------------ align: 2
    LATEX \FloatBarrier
    SUBSECTION Aligning equations in multiple columns

    TEXT Suppose you wanted to demonstrate three kinds of derivative: polynomial, trigonometric and exponential. And you wanted to do so in minimal vertical space. Then you might typeset something like \cref{ex:math-align-multi-1}.

    LBTEXAMPLE .o vertical
    :: (label) ex:math-align-multi-1
    :: (caption) Alignment in multiple columns
    :: .v <<
      MATH* .o align
      :: f(x)  &= x^3 - 7x^2 + 4x + 1  & g(x)  &= sin(2x) - tan x  & h(x)  &= 3^x
      :: f'(x) &= 6x^2 - 14x + 4       & g'(x) &= 2cos(2x) - sec2x & h'(x) &= 3^x\:ln 3
    >>

    TEXT The space between the \qq{columns} is determined by the ◊amsmath package -- see the relevant documentation for details. If you want to really spread things out, you can use ◊flalign, which uses the \qq{full length} of the page, as shown in \cref{ex:math-align-multi-2}.

    LBTEXAMPLE .o vertical
    :: (label) ex:math-align-multi-2
    :: (caption) Full-length in multiple columns
    :: .v <<
      MATH* .o flalign
      :: f(x)  &= x^3 - 7x^2 + 4x + 1  & g(x)  &= sin(2x) - tan x  & h(x)  &= 3^x
      :: f'(x) &= 6x^2 - 14x + 4       & g'(x) &= 2cos(2x) - sec2x & h'(x) &= 3^x\:ln 3
    >>

    TEXT And if you want to determine your own spacing, you can: the ◊alignat environment gives the author that control. \cref{ex:math-align-multi-3} demonstrates the use of \Verb|\qquad| to separate the two columns. When you use ◊alignat, you need to provide the oparg \code{ncols} to specify how many columns there are.\footnote{Note that in this example, there are two columns and $2(2)-1=3$ ampersands per line. It is helpful to keep this relationship in mind.}

    LBTEXAMPLE .o vertical
    :: (label) ex:math-align-multi-3
    :: (caption) Manual control of inter-column spacing
    :: .v <<
      MATH* .o alignat, ncols = 2
      :: f(x)  &= x^3 - 7x^2 + 4x + 1  &\hspace{4em} g(x)  &= sin(2x) - tan x
      :: f'(x) &= 6x^2 - 14x + 4       &             g'(x) &= 2cos(2x) - sec2x
    >>

    TEXT Generally speaking, the default spacing should be sufficient. A more useful purpose for ◊alignat is shown in \emph{\nameref{subsec:multi-points}} on \cpageref{subsec:multi-points}, where polynomials have their like terms lined up regardless of the width of coefficients.

    % ------------------------------------------------------------ align: 3
    LATEX \FloatBarrier
    SUBSECTION Improving the display of a single long equation

    TEXT Earlier, we saw how ◊multiline can be used to manually break up an equation that doesn't fit on one line. The mixed-justification formatting that ◊multiline applies may suit some equations but not others. If you prefer a left-justified equation as shown in \cref{ex:math-align-multiline-1}, this can be achieved with ◊alignat and a single column. Note the use of \Verb|\MoveEqLeft| from the \package{mathtools} package to place the first line correctly. Also note the use of \Verb|\phantom{=\ }| to align the continuation lines nicely.

    LBTEXAMPLE .o vertical
    :: (label) ex:math-align-multiline-1
    :: (caption) Using ◊alignat and \code{MoveEqLeft} to improve the formatting of a long equation
    :: (shortcaption) Using ◊alignat and \code{MoveEqLeft} for a long equation
    :: .v <<
      STO ph :: 1 :: \phantom{=\ }
      MATH .o alignat, ncols = 1, eqnum = 5
      :: \MoveEqLeft (x - r_1)(x - r_2)(x - r_3)(x - r_4)
      :: quad &= x^4 - (r_1 + r_2 + r_3 + r_4)x^3
      ::      &◊ph + (r_1r_2 + r_1r_3 + r_1r_4 + r_2r_3 + r_2r_4 + r_3r_4)x^2
      ::      &◊ph - (r_1r_2r_3 + r_1r_2r_4 + r_1r_3r_4 + r_2r_3r_4)x
      ::      &◊ph + r_1r_2r_3r_4
    >>


    % ------------------------------------------------------------ align: 4
    LATEX \FloatBarrier
    SUBSECTION Providing comments to the right

    TEXT It is common that an author wants to write some brief commentary to the right of a line of working in a multi-step equation. We can achieve this using ◊align with two columns, as \cref{ex:math-align-commentary-1} demonstrates.

    LBTEXAMPLE .o vertical, position = htbp
    :: (label) ex:math-align-commentary-1
    :: (caption) Using \code{MATH .o align} for a multi-step equation with commentary
    :: .v <<
      TEXT Part of a proof by induction.
      STO half :: 2 :: $\tfrac 1 2$
      MATH* .o align
      :: 2^{n+1} &= 2 cdot 2^n
      ::         &> 2 cdot n^2                    && text {(by assumption)}
      ::         &= n^2 + ◊half n^2 + ◊half n^2
      ::         &> n^2 + 2n + 1                  && text {(reader to confirm)} \tag{*}
      ::         &= (n+1)^2

      TEXT The reader who is interested in tackling (*) might like to consider how we know that $◊half n^2 > 2n$ and $◊half n^2 > 1$.
    >>

    TEXT The inter-column spacing in \cref{ex:math-align-commentary-1} is too large, so we assert manual control using ◊alignat, as shown in \cref{ex:math-align-commentary-2}. Note that we insert a \Verb|\qquad| in the \emph{longest} line of working.

    LBTEXAMPLE .o vertical, position = htbp
    :: (label) ex:math-align-commentary-2
    :: (caption) Using \code{MATH .o alignat} to improve the previous example
    :: .v <<
      TEXT Part of a proof by induction.
      STO half :: 2 :: $\tfrac 1 2$
      MATH* .o alignat, ncols = 2
      :: 2^{n+1} &= 2 cdot 2^n
      ::         &> 2 cdot n^2                   && text {(by assumption)}
      ::         &= n^2 + ◊half n^2 + ◊half n^2  \qquad && text {}
      ::         &> n^2 + 2n + 1                 && text {(reader to confirm)} \tag{*}
      ::         &= (n+1)^2

      TEXT The reader who\dots
    >>

    TEXT
    :: In these examples we have been typesetting a single multi-step equation (for which ◊split is designed) using ◊align, which is designed for multiple logical equations. It is perhaps a shame that the ◊amsmath package does not support this use-case---providing commentary on a split equation---more directly.
    :: Having said that, there is the option to use ◊alignedat, instead of ◊alignat, and place that inside ◊equation. ◊alignedat does the logical layout without doing any numbering, and ◊equation displays the result and assigns a number. ◊lbt supports this combination with ◊eqalignedat, as shown in \cref{ex:math-align-commentary-3}.

    LBTEXAMPLE .o vertical, position = htbp
    :: (label) ex:math-align-commentary-3
    :: (caption) Using \code{MATH .o eqalignedat} to align a single logical equation
    :: .v <<
      TEXT Part of a proof by induction.
      STO half :: 1 :: $\tfrac 1 2$
      MATH .o eqalignedat, ncols = 2, label = eq:induc
      :: 2^{n+1} &= 2 cdot 2^n
      ::         &> 2 cdot n^2                   && text {(by assumption)}
      ::         &= n^2 + ◊half n^2 + ◊half n^2  \qquad && text {}
      ::         &> n^2 + 2n + 1                 && text {(reader to confirm)}
      ::         &= (n+1)^2

      TEXT The techniques in \eqref{eq:induc} should be mastered by all students.
    >>

    % ------------------------------------------------------------ align: 5
    LATEX \FloatBarrier
    SUBSECTION Incorporating lines of text

    TEXT Sometimes you want to interrupt a sequence of aligned equations to provide a line or even paragraph of commentary. This is achieved with the ◊amsmath command ◊intertext or (if tighter vertical spacing is desired) the ◊mathtools command ◊shortintertext. In ◊lbt you can use these as you would in plain Latex, or you can use the more ◊lbt-ish versions \code{TEXT} and \code{SHORTTEXT}. \cref{ex:math-align-intertext-1} demonstrates.

    CTRL microdebug :: on
    LBTEXAMPLE .o vertical
    :: (label) ex:math-align-intertext
    :: (caption) Including a line of text amidst an aligned block
    :: .v <<
      MATH .o align, eqnum = 3 5
      ::         S_n &= a + ar + ar^2 + dots + ar^{n-1}
      :: TEXT Multiply both sides by $r\,$:
      ::       r S_n &= ar + ar^2 + ar^3 + dots + ar^n
      :: TEXT Now subtract the second equation from the first:
      :: S_n - r S_n &= (a + ar + ar^2 + dots + ar^{n-1}) - (ar + ar^2 + dots + ar^n)
      :: TEXT All terms cancel except the first and the last:
      :: (1 - r) S_n &= a - ar^n
      :: TEXT Finally, divide both sides by $1-r$ (assuming $r \neq 1$):
      ::         S_n &= frac {a(1 - r^n)} {1 - r}
    >>

    % ------------------------------------------------------------ align: 6
    LATEX \FloatBarrier
    SUBSECTION (label) subsec:multi-points :: Multiple alignment points among equations

    % ------------------------------------------------------------ align: 7
    LATEX \FloatBarrier
    SUBSECTION Aligning equations near the left margin

    % ------------------------------------------------------------ Other environments
    LATEX \FloatBarrier
    SECTION (label) sec:math-other :: \code{Other environments}

    SUBSECTION ◊split
    TEXT
    :: \textbf{Revisit this text in light of it being in the \qq{other} section}
    :: ◊MATH provides the \code{split} option to access the ◊split environment, but it is not likely to be all that useful, because of the need to enclose it in another environment. The example below shows the ◊lbt code and resulting Latex code.

    LBTEXAMPLE .o float = false, horizontal, output = 1, shrinkmargin = nil
    :: .v <<
      MATH .o split
      :: (a+b)^2 &= (a+b)(a+b)
      ::         &= a^2 + ab + ab + b^2
      ::         &= a^2 + 2ab + b^2
    >>

    % ------------------------------------------------------------ Additional environments
    LATEX \FloatBarrier
    SECTION (label) sec:math-other :: \code{Additional environments}

    TEXT
    :: ◊MATH's main \emph{raison d'être} is to provide convenient ◊lbt-style access to the ◊amsmath and ◊mathtools environments. However, two use cases demonstrated in \cref{ex:math-align-multiline-1} (allowing an equation to \qq{continue} over multiple lines) and \cref{ex:math-align-commentary-3} (providing neatly aligned \qq{commentary} to the right) have some boilerplate code that could be abstracted away. Thus ◊MATH provides the ◊continuation and ◊commentary psuedo-environments.
    :: The two examples cited above are reproduced inline in the subsections below, for convenience, to provide an easy comparison between the ◊amsmath-style approach and the ◊lbt-style approach.

    SUBSECTION ◊continuation

    TEXT \cref{ex:math-align-multiline-1} demonstrated a long equation continuing over several lines with left justification and attention to alignment. The example is reproduced below for convenvience.

    LBTEXAMPLE .o vertical, float = false
    :: .v <<
      STO ph :: 1 :: \phantom{=\ }
      MATH .o alignat, ncols = 1, eqnum = 5
      :: \MoveEqLeft (x - r_1)(x - r_2)(x - r_3)(x - r_4)
      :: quad &= x^4 - (r_1 + r_2 + r_3 + r_4)x^3
      ::      &◊ph + (r_1r_2 + r_1r_3 + r_1r_4 + r_2r_3 + r_2r_4 + r_3r_4)x^2
      ::      &◊ph - (r_1r_2r_3 + r_1r_2r_4 + r_1r_3r_4 + r_2r_3r_4)x
      ::      &◊ph + r_1r_2r_3r_4
    >>

    TEXT ◊MATH provides a ◊continuation option that allows the same result to be achieved more easily, relieving authors of the need to be experts in \code{phantom} and \code{MoveEqLeft}. It is demonstrated in \cref{ex:math-continuation}.

    LBTEXAMPLE .o vertical
    :: (label) ex:math-continuation
    :: (caption) ◊continuation supporting neat display of a long equation
    :: .v <<
      MATH .o continuation
      :: (x - r_1)(x - r_2)(x - r_3)(x - r_4)
      :: = x^4 - (r_1 + r_2 + r_3 + r_4)x^3
      ::   + (r_1r_2 + r_1r_3 + r_1r_4 + r_2r_3 + r_2r_4 + r_3r_4)x^2
      ::   - (r_1r_2r_3 + r_1r_2r_4 + r_1r_3r_4 + r_2r_3r_4)x
      ::   + r_1r_2r_3r_4
    >>

    TEXT The code produced behind the scenes is the same as the code demonstrated above, using ◊alignat and \code{MoveEqLeft} and \code{phantom}.

    % ------------------------------------------------------------ Combinations
    LATEX \FloatBarrier
    SECTION (label) sec:combinations :: Combinations



    % ------------------------------------------------------------ Summary
    LATEX \FloatBarrier
    SECTION \code{Summary of the \code{MATH} command}





\end{lbt}
