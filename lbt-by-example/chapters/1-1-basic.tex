\begin{lbt}
  !DEBUG
  [@META]
    TEMPLATE lbt.Doc.Chapter
    TITLE Core commands
    LABEL ch.basic
    SOURCES LbtDoc

  [+BODY]
    TEXT The \package{lbt.Basic} template implements:
    ITEMIZE .o compact
    :: document divisions (part, chapter, section, subsection, subsubsection, paragraph, subparagraph)
    :: low-level typesetting facilities (vertical space, arbitrary commands and environments, flushleft, flushright, center)
    :: things you expect in normal Latex editing, whether built in or using a common plugin (columns, verbatim, various math environments)
    :: things that are generally useful (include PDF pages, three levels of headings, place two items side-by-side)

    TEXT It also implements lists and tables, which are demonstrated in \cref{ch:liststables}.

    % ------------------------------------------------------------ Document divisions
    SECTION Document divisions

    TEXT All the Latex commands are present. We present \code{CHAPTER} and \code{SECTION} in \cref{ex:basic-divisions}, but don't show the typeset results as we don't want to affect the chapter of this book!

    LBTEXAMPLE .o vertical, output = 1
    :: (caption) Chapters, sections and plain text paragraphs
    :: (label) ex:basic-divisions
    :: .v <<
      CHAPTER (label) ch:canopy :: Beneath the canopy

      TEXT When we first visited the rainforest, \dots

      SECTION (label) sec:life-leaves :: Life through the leaves

      TEXT From the forest floor, you can't see any direct light at all \dots
    >>

    % --------------------------------------- Ordinary text and whitespace
    SECTION Ordinary text and whitespace

    TEXT \Cref{ex:basic-text} shows the \code{TEXT} command, which outputs one or more paragraphs. \code{TEXT*} suppresses the (final) \verb|\par|. You can input vertical space with \code{VSPACE}. Further, all commands accept optional arguments \code{pre} and \code{post} for including some surrounding whitespace.

    LBTEXAMPLE .o vertical
    :: (caption) \code{TEXT} (single and multiple paragraphs) and \code{VSPACE}
    :: (label) ex:basic-text
    :: .v <<
      TEXT Can you guess which author wrote these sentences about chess?
      VSPACE 1em
      TEXT
      :: The chessboard is the world; the pieces are the phenomena of the universe; the rules of the game are what we call the laws of Nature.
      :: The beauty of a move lies not in its appearance but in the thought behind it.
      :: In chess, as in life, a man is his own most dangerous opponent.
      TEXT* .o pre = 2em :: It was\dots
      TEXT an early world champion.
    >>

    TEXT Other low-level formatting commands include \code{CLEARPAGE} and \code{VFILL}, both of which are hard to demonstrate, but predictable. Finally, there is \code{VSTRETCH}, which helps spread items out vertically. For example, \cref{ex:basic-vstretch} spreads out three (very short) paragraphs to fill a page, and allocates more whitespace in the middle than the other two.

    LBTEXAMPLE .o float, output = 0
    :: (caption) Using \code{VSTRETCH} to spread out paragraphs on a page
    :: (label) ex:basic-vstretch
    :: .v <<
      TEXT Top paragraph.
      VSTRETCH 1
      TEXT Middle paragraph.
      VSTRETCH 1.5
      TEXT Bottom paragraph.
      VSTRETCH 1
      CLEARPAGE
    >>

    TEXT The \code{CLEARPAGE} is necessary to ensure that all vertical space on the page is used.





    % ----------------------------------------------- Margins and justification
    SECTION Margins and justification

    TEXT \cref{ex:basic-margins} demonstrates \code{INDENT}, which adjusts the left and right margin using the \code{adjustwidth} environment from the \code{changepage} package. The first argument defines the inward margin for left and right margins. (Note that the right margin adjustment defaults to zero.) The second argument is the text.

    LBTEXAMPLE .o vertical
    :: (caption) Using \code{INDENT} to adjust left and (optionally) right margins
    :: (label) ex:basic-margins
    :: .v <<
      INDENT 4cm, 2cm :: It is a truth universally acknowledged, that a single man in possession of a good fortune, must be in want of a wife.

      INDENT 6cm :: However little known the feelings or views of such a man may be on his first entering a neighbourhood, this truth is so well fixed in the minds of the surrounding families, that he is considered as the rightful property of some one or other of their daughters.
    >>

    TEXT \cref{ex:basic-justification} demonstrates \code{FLUSHLEFT}, \code{FLUSHRIGHT} and \code{CENTER}, which affect the justification of the contained paragraph(s). Incidentally, it also demonstrates \code{STO}, which saves (\qq{stores}) some content for a limited amount of time. For more information about \code{STO}, see \cref{ex:sto}.

    LBTEXAMPLE .o vertical
    :: (caption) Left, center and right justification
    :: (label) ex:basic-justification
    :: .v <<
      STO text :: 3 :: However little known the feelings or views of such a man may be on his first entering a neighbourhood, this truth is so well fixed in the minds of the surrounding families, that he is considered as the rightful property of some one or other of their daughters.

      FLUSHLEFT ◊text
      CENTER .o pre = 1ex :: ◊text
      FLUSHRIGHT .o pre = 1ex :: ◊text
    >>

    STO sep :: 1 :: {:}{:}
    NOTE The examples show that \code{INDENT} and \code{CENTER}, et cetera, operate only on the paragraph(s) given to them. This raises a question: how do you center, say, a whole block of \lbtlogo{} code? \par There are two answers. First, you could manually invoke the \code{center} environment with \code{BEGIN center}, then place your code, then \code{END center}. This is low-level and not in the spirit of \lbtlogo. The second option is to put all your content in a register with \code{STO .o lbt ◊sep content ◊sep .v << ... >>}, and then use \code{CENTER ◊content}. \par \lbtTodo{For this documentation, readers should be directed to a section that demonstrates STO in detail.}

\end{lbt}
