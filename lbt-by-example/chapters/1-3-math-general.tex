\begin{lbt}
  !NODRAFT
  [@META]
    TEMPLATE lbt.Doc.Chapter
    TITLE Mathematical text \textsf{(lbt.Math)} -- various macros
    LABEL ch.math-macros
    SOURCES LbtDoc

  [+BODY]
    STO simplemath :: 999 :: \code{simplemath}
    STO lbt :: 999 :: \lbtlogo{}

    TEXT The \package{lbt.Math} template provides several affordances for typing mathematical text.
    ITEMIZE .o compact
    :: The ◊simplemath macro is a replacement for the inline and display math environments \Verb|$ ... $| and \Verb|$$ .. $$| (or their Latex equivalents). It allows you to type mathematical text more succinctly and with fewer backslashes.
    :: The \code{integal} macro simplifies typing definite and indefinite (single) integrals.
    :: The \code{vector} and \code{vectorijk} macro greatly simplify typing vectors.
    :: A collection of macros like \code{mathlistand}, \code{mathsum}, \code{mathseq} and several others provide a convenient way to type mathematical text like \qq{\mathlistand{a,b,c}} or \qq{\mathseq{y,1,2,3,...,n}}.

    TEXT Also found in \package{lbt.Math} is the \code{MATH} command, which aids in the typesetting of display equations. It appears in \cref{ch.math-command}.

    % ------------------------------------------------------------ The simplemath macro
    SECTION The ◊simplemath macro

    BOX :: \verb|\lbtDefineMacros{sm = lbt.Math:simplemath}|

    TEXT
    :: The ◊simplemath macro stands in for \Verb|$ ... $| or \Verb|$$ .. $$| and allows your to type mathematical text succinctly. It recognises a lot of keywords and abbreviations, meaning far fewer backslashes are needed.
    :: \cref{ex:math-sm-1} demonstrates the use of ◊simplemath in both math modes: inline and display. \cref{ex:math-sm-table} contains a large number of examples to show the variety of conveniences that ◊simplemath offers.
    :: \cref{ex:math-sm-2} shows that brackets, both round and square, are auto-sized (that is, \Verb|\left| and \Verb|\right| are applied intelligently) and that \Verb|text{...}| (or \Verb|\text{...}|) is recognised specially, and passed through to Latex without processing its contents. Note, however, that braces (\Verb|{}|) are \emph{not} auto-sized.

    LBTEXAMPLE .o vertical
    :: (caption) ◊simplemath usage, both inline and display form
    :: (label) ex:math-sm-1
    :: (breakat) BREAK
    :: .v <<
      TEXT Two interesting results from Euler are BREAK \sm{ds arctan x = x - frac{x^3}3 + frac{x^5}5 - cdots} and BREAK \sm{ sum_{n=1}^{infty} frac 1 {n^2} = frac {pi^2} 6. }
    >>

    EXAMPLETABLE .o wrapcell = sm
    :: (label) ex:math-sm-table
    :: (caption) A collection of ◊simplemath examples
    :: cos2 th + sin2 th equiv 1
    :: log_2 n ge 5
    :: ds (1+x)^n = sum_{r=0}^n binom n r x^r
    :: A = B iff A subseteq B vee B subseteq A
    :: ds al be + al ga + be ga = frac {-b} {2a}
    :: ds f'(x) = lim_{h to 0} frac {f(x+h)-f(x)} h
    :: neg(P implies Q) equiv P vee neg Q
    :: P imp Q equiv neg P wedge Q
    :: 12 = 2 cdot 2 cdot 3 text{ so } 5 nmid 12
    :: ds prod_{i=1}^n x_i = x_1 x_2 cdots x_n
    :: sqrt 2 notin bbQ
    :: forall n in bbZ, n^2 in bbZ
    :: OABC text{ is a parallelogram.}


    LBTEXAMPLE .o vertical
    :: (label) ex:math-sm-2
    :: (caption) ◊simplemath's handling of parentheses, brackets and \code{text}
    :: (breakat) BREAK
    :: .v <<
      TEXT Brackets will automatically assume the correct size: BREAK
      » \sm{ y = (1 + [frac x 7])^2 quad text{($[A]$ is the rounding function)} }

      TEXT Suppress this behaviour by setting the option \texttt{simplemath.leftright = false}.
    >>

    % ------------------------------------------------------------ The integral macro
    LATEX \FloatBarrier
    SECTION The \code{integral} macro

    TEXT
    :: Typing integrals in regular Latex, or with ◊simplemath, is not exactly a chore. But there is room for simplification, and the \code{integral} macro allows for definite or easy typing indefinite (single, simple) integrals. If the first argument is \code{ds} then a \Verb|\displaystyle| is inserted.
    :: The resulting Latex code is wrapped in \Verb|\ensuremath{...}|, so integrals do not need to be inside a math environment.
    :: \cref{ex:math-int-1} shows a definite and indefinite integral in context. \cref{ex:math-int-2} contains enough examples so that definite and indefinte are shown, as are normal style and display style.

    LBTEXAMPLE .o vertical
    :: (caption) The \code{integral} macro
    :: (label) ex:math-int-1
    :: (breakat) BREAK
    :: .v <<
      TEXT The integral \integral{ds,\frac{\sin x}{x},dx} has practical importance but can't be evaluated in closed form.

      TEXT You can take advantage of ◊simplemath as well: BREAK \sm{ \integral{pi/3,pi,sqrt {1 + sin3 th},d th} }
    >>

    EXAMPLETABLE
    :: (label) ex:math-int-2
    :: (caption) A collection of \code{integral} examples
    :: \integral{\sin x,dx}
    :: \integral{1,3,\sqrt{4z},dz}
    :: \integral{ds,\tan y,dy}
    :: \integral{ds,1,3,\sqrt{4z},dz}



    % ------------------------------------------------------------ The vector macro and friends
    LATEX \FloatBarrier
    SECTION The \code{vector} macro

    TEXT
    :: With the \code{vector} macro you can easily typeset \vecbold{a} and \vectilde{b} and \vecarrow{c}. Oh, and \V{DE} and \V{3 -7} and \V{row 4 0 9}. And finally, \V{ijk 2 -5 1} and \V{ijk -3 0 4}.
    :: One might choose to set up this macro as \Verb|\V| as follows:
    BOX \verb|\lbtDefineMacros{V = lbt.Math:vector}|
    TEXT A thorough set of examples is given in \cref{ex:math-vector-table}.

    EXAMPLETABLE .o mathmode
    :: (label) ex:math-vector-table
    :: (caption) A collection of \code{vector} and \code{vectorijk} examples
    :: \V{3 1 -9} + \V{-2 5 4} = \V{1 6 -5}
    :: \V{row 2 7 -1 4 6}
    :: \V{a} + \V{b} = \V{a_1 a_2} + \V{b_1 b_2}
    :: \V{r} = \V{r_1 \vdots r_n}
    :: \V{3 1 -9} = \Vijk{3 1 -9}
    :: \V{-1 0 4} = \Vijk{-1 0 4}
    :: \V{-1 0} = \Vijk{-1 0}
    :: \V{ijk 3 4 5}

    SUBSECTION Bold, tilde and arrow
    TEXT Vectors are most commonly set in bold upright (\vecbold{a}), and that is the \lbtlogo{} default. You can, however, choose tilde (\vectilde{a}) or arrow (\vecarrow{a}) instead. \cref{ex:math-vector-formats-1} shows how to make these choices for your \lbtlogo{} expansion or your whole Latex document. If you want just a one-off, you can define and use the following macros. \cref{ex:math-vector-formats-2} demonstrates their use.
    BOX
    :: \verb|\lbtDefineMacros{vecbold  = lbt.Math:vecbold}|
    :: \verb|\lbtDefineMacros{vectilde = lbt.Math:vectilde}|
    :: \verb|\lbtDefineMacros{vecarrow = lbt.Math:vecarrow}|

    LBTEXAMPLE .o output = 0
    :: (caption) Vectors in bold or tilde or arrow format, document-wide
    :: (label) ex:math-vector-formats-1
    :: (substitute) LBT/lbt
    :: .v <<
      * Affect the whole Latex document
      \lbtGlobalOpargs{vector.format = tilde}

      * Affect one LBT expansion
      \begin{LBT}
        [@META]
          TEMPLATE lbt.Basic
          OPTIONS  vector.format = arrow
        ...
      \end{LBT}
    >>

    LBTEXAMPLE .o vertical
    :: (caption) Vectors in bold or tilde or arrow format, one-off
    :: (label) ex:math-vector-formats-2
    :: .v <<
      TEXT We can write \vecbold{a} or \vectilde{b} or \vecarrow{c}.
    >>

    % ---------------------------------------------------- The mathlistand macro and friends
    LATEX \FloatBarrier
    SECTION List, sequence, summation and product macros

    TEXT
    :: It is frequently necessary to type in a collection of numeric or algebraic terms expressed as a list or a sum or a product. Sometimes the terms are related, like $a_1$, $a_2$ etc. Typing such things into Latex can be a minor annoyance if each term is wrapped in dollar signs. There is repeated structure that can be abstracted by a macro, especially with a proper programming language available.
    :: The table below shows some examples of each, with the Latex used to achieve them.

    NEWCOMMAND myverb :: 1 :: \texttt{\detokenize{#1}}

    TABLE .o center, fontsize = small
    :: (spec) colspec = {lll}, row{1} = {font=\bfseries}
    :: Desired output & Latex code
    :: \midrule
    :: \mathlist{a,b,c}             & \myverb{$a, b, c$}
    :: \mathlistand{13,14,15,16}    & \myverb{$13$, $14$, $15$, and $16$}
    :: \mathlistor{X,Y,Z}           & \myverb{$X$, $Y$, or $Z$}
    :: \mathlist{a,b,c,...,z}       & \myverb{$a, b, c, \dots, z$}
    :: \mathsum{1,2,3,...,n}        & \myverb{$1 + 2 + 3 + \dots + n$}
    :: \mathseq{T,4,5,6}            & \myverb{$T_4, T_5, T_6$}
    :: \mathseqsum{T,4,5,6}         & \myverb{$T_4 + T_5 + T_6$}
    :: \mathseqproduct{a,1,2,...,N} & \myverb{$a_1 a_2 \dots a_N$}
    :: \mathseqproductcdot{a,1,2,...,N} & \myverb{$a_1 \cdot a_2 \cdot \dots \cdot a_N$}

    TEXT
    :: ◊lbt offers some macros to make entering such things more convenient, which are demonstrated in \cref{ex:math-sequence-macros}. See \cref{sec:math-set-up-macros} for code to include in your preamble to gain access to these macros.

    EXAMPLETABLE
    :: (label) ex:math-sequence-macros
    :: (caption) List/sequence macros such as \code{mathsum} and \code{mathseqproduct}
    :: \mathlist{m,n,p,q,r}
    :: \mathlistand{a,b,c,d}
    :: \mathlistor{X,Y,Z}
    :: \mathlist{3,4,5,...,9}
    :: \mathlist{2^0,2^1,2^2,...,2^{n-1},2^n}
    :: \mathsum{a,b,c,d,e}
    :: \mathsum{p_1,p_2,p_3,...,p_n}
    :: \mathproductcdot{p_1,p_2,p_3,...,p_n}
    :: \mathseq{T,1,2,3,4,5}
    :: \mathseqsum{T,1,2,3,4,5}
    :: \mathseqproduct{T,1,2,3,4,5}
    :: \mathseq{p,1,2,3,...,n}
    :: \mathseqsum{p,22,23,24,...,29,30}
    :: \mathseqproduct{p,1,2,...,5}
    :: \mathseqproductcdot{p,1,2,...,5}

    LATEX \FloatBarrier
    SECTION Miscellaneous macros

    TEXT
    :: ◊lbt offers some other macros that bring convenience to certain tasks in mathematical typesetting.
    :: Currently the only such macro is \code{primefactorisation}, which is demonstrated in \cref{ex:math-primefac}.

    EXAMPLETABLE
    :: (label) ex:math-primefac
    :: (caption) Miscellaneous macros such as \code{primefactorisation}
    :: \primefactorisation{2,2,2,5,7,7,19}
    :: \primefactorisation{explicit 2,2,2,5,7,7,19}


    % ------------------------------------------------------------ Code to set up all macros
    LATEX \FloatBarrier
    SECTION (label) sec:math-set-up-macros :: Code to set up all macros

    TEXT
    :: As a convenience, \cref{ex:math-macros-setup} contains the code that you can paste into your preamble to obtain access to all macros described in this chapter. They are grouped for readability.

    LBTEXAMPLE .o output = 0
    :: (label) ex:math-macros-setup
    :: (caption) Preamble code to set up all \package{lbt.Math} macros
    :: .v <<
      \lbtDefineLatexMacros{
        sm                 = lbt.Math:simplemath,
        integral           = lbt.Math:integral,
      }
      \lbtDefineLatexMacros{
        V                  = lbt.Math:vector,
        vecbold            = lbt.Math:vecbold,
        vecarrow           = lbt.Math:vecarrow,
        vectilde           = lbt.Math:vectilde,
      }
      \lbtDefineLatexMacros{
        mathlist           = lbt.Math:mathlist,
        mathlistand        = lbt.Math:mathlistand,
        mathlistor         = lbt.Math:mathlistor,
        mathsum            = lbt.Math:mathsum,
        mathproductcdot    = lbt.Math:mathproductcdot,
        mathseq            = lbt.Math:mathseq,
        mathseqsum         = lbt.Math:mathseqsum,
        mathseqproduct     = lbt.Math:mathseqproduct,
        mathseqproductcdot = lbt.Math:mathseqproductcdot,
      }
      \lbtDefineLatexMacros{
        primefactorisation = lbt.Math:primefactorisation,
      }
    >>




\end{lbt}
